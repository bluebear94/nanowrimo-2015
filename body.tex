\chapter{The Case of the Night-Runner}

\emph{Note: this work uses the convention of placing the surname before the given name.}

As the dim moonlight forspreaded through the ever-waking boughs of cedars rustling in a gentle lullaby among the placid wind, and the nearby creek, once a leviathan waterfall from the distant peaks, flowed as it always did, a young person with flowing lavender hair in open clothing sat on the birch-plank floor of the fifth-story loggia, enjoying a rare hour free of mundane housework. Cytaen Myllada, no more than thirteen years of age at that time, once could fantasize of tales and legends, war and peace, the great and the small, that lay beyond the cedars, or even this village, and had even written of them, but now her imaginations had nearly run out. After all, speculating what is there does not compare to actually knowing what is there, nor does mere knowledge compare to seeing these places firsthand.

Myllada also held a long-standing interest over mathematics and nature, interests that her parents, allegedly desiring an ignoramus to perform chores and marry young, would not tolerate. Having seen no interior of a school or library, she had learned to read using a book she had received eight years ago from an eccentric instructor, stashed in a closet alongside seven score pages of her own writing, avoiding the wrath of the flame.

By then, this inquisitiveness had vexed the young person for no fewer than three thousand nights in the confines of the mansion uneroded by time, wherein one could not find any traces of modern society.

The unexpected clamor of bushes snapped her out of reverie. Myllada, who seldom had the opportunity of conversing with those outside her family, transitioned into a kneeling position, peering over the rails in order to discern who, or what, made that sound.

Flashing a cloak as dark as the rest of its outfit, a form scurried along the side of the building.

Loudly enough for the figure to hear the sound, but quietly enough to avoid waking up anyone else in the house, Myllada whistled.

The figure stopped and turned around, running to the source of the whistle.

Knowing that any speech would attract attention, Myllada brandished her hand, receded into her room, pulled out a sheet of paper and a pen, and scribbled a message before folding the paper several times and tossing it onto the needle-covered grounds, ensuring that the letter did not land on a lower loggia.

The note thudded on hitting the ground, at which the unidentified scampered to collect it, before bowing shallowly and making haste out of the forest.

As the surroundings of the house again became desolate, Myllada again felt the hopelessness of residing there. But with this cloud came a silver lining: the narrowest probability that whomever she spotted would respond to her plea; help her escape the wretched prison.

\centeredstars

As the sun rose, a person, having returned home not long ago, opened a note that flew from a mansion untouched by centuries past. Not only was the letter unsigned, but the handwriting flew in all directions, hardly organized, and a style unexpected from someone who would live in such a place.

\q{
	I have been stuck in this house for many years, without a day of school, being raised as an ignoramus whose only purpose is to maintain the house and marry young. You are one of the few people other than family members whom I can reach.
	
	Please consider replying by any feasible measures. If I find you helpful, then I will find something for you in return. Make sure to arrive at night and bring clothing.
}

The recipient wondered why the sender asked for clothing -- perhaps the sender was naked? In any case, there was work to be done, to which the message was completely irrelevant.

\centeredstars

By midday the excitement over the mysterious visitor has eroded, and the possibility of any further external contact seemed a distant dream. \emph{That person will never come; what do I have in return?} Myllada pondered as she swept the corridors, lapsing for a few seconds.

``A woman who does not sweep is worthless!'' came a deep yell behind her, shocking her back into the monotonous task.

\emph{I could count the people who use that word with one hand. Well, on the bright side, if I'm worthless, then it shouldn't matter much to get out of here!} Myllada humorously concluded while joining thought and movement into the same rhythm. \emph{But, of course, I can't just brazenly say ``I'm useless to you'' and frolic out through the door.}

\centeredstars

Once again the moonlight diffused through the tree-branches and save the ever-running creek and the whispering of the trees all was quiet. Myllada was in her room tonight, lying in bed, when she heard the familiar clamor of bushes and opened the door to the loggia in response.

It appeared to be the same night-walker, now peering upwards toward her vicinity. The figure observed the trees, as if it were finding a path into the house, but after a long moment, it left behind a large sack and ran away into the depths of the woods.

\emph{Whatever it left... it must be coming back, but for what?} pondered Myllada as she receded to her room. She opened the closet door and pulled out a stack of papers hidden behind the immodest clothing she had. Kneeling toward the moonlight, she picked up random pages to read and held them to the air.

\q{
	Sturdy wooden bridges span the mountains between which the river was born, connecting the buildings suspended in the air. Most of the rooms are open to the cold air, with tall windows with no glass.
	
	Near the valleys tunnels a \emph{veten} wide are bored in the walls of the mountains, leading to mineshafts reaching up to three \emph{navso} underground.
}

And another:

\q{
	In the forests to the south there once existed a great city, two \emph{navso} across, whose tallest structures could be observed from eighty \emph{navso} away. One who traveled through it would be confused about whether it was day or night.
	
	It has been long since it was flattened to ruins, but several visitors still frequent a mysterious shack near its center --
}

Hearing a whistle, Myllada opened the door to the loggia and noticed a contraption brought by the night-runner -- what seemed to be a tray with a crank to the side.

Upon depositing the sack onto the tray, the figure turned the crank, causing the machine to lift the tray up to the fifth floor using some sort of metallic arm.

Myllada retrieved and opened the sack, finding a pair of sandals and a plum-colored robe. Inside the robe was the note she had sent, with a message on the back: \emph{Can you get out?} to which she, still in the open dress, and discerning that the platform obviously could not support a human being, took out a pen and replied:

\q{
	All of the doors are sealed. Perhaps I could jump off, but how would I return?
}

\chapter{The Tailless Long-Tail}

When Myllada rose again in the morning, she felt that she had not slept enough, but her father was yelling for her in the voice she had learned to abhor. Her name, her appearance, her voice -- she detested nearly everything in her current life, but now was the wrong time for absentminded mentation, and she reluctantly stood up and opened the door to the rest of the edifice.

At the table over the early meals, always eaten in sleeping-clothes, her father inquired, ``Thou lookest half-asleep. Hast thou been awake all night?''

``I have not,'' replied Myllada, while sipping her tea. The taste, she noticed, was slightly exotic.

\centeredstars

Another night fell and Myllada was alone in her room, putting on the clothing that she received the past night and anticipating another visit, reading her old compositions to keep herself occupied.

Suddenly, she felt exhausted and could not concentrate on the passages. A desire to sleep encroached on her mind, more influential than the anticipation of a friend, and she, still covered to the feet and down the arms, unfolded her body and laid her head on her right arm, allowing a curtain of darkness to descend.

\centeredstars

Myllada woke up, face in a bucket of water. To the side there was a person no more than four and twenty years old, with short dark brown hair, as well as a lantern, which was somehow lit without a flame.

Rising into a sitting position, Myllada asked, ``How did you get in?''

``One of the windows wasn't sealed properly. My name is Darmjarel Telto.''

``Cytaen Myllada. Where's your tail? I can't find it.''

``I'd like to look at your necklace. It looks quite attractive on you.''

Myllada pulled out a fine golden chain. The red gem at the front was cut into a rhombus, but there was an orange mist trapped inside it.

Telto almost jumped in surprise. ``This is a \emph{tracking charm}. Long since they've been banned. As long as it's on, whoever put it there knows where you are. Severing the chain isn't of any use either; that person will be notified with immense pain. Now if you could find where the two ends of the chain were connected, you could disconnect them and avoid these measures. However, the necklace is hot to the touch for anyone but the wearer.''

``Now, how difficult is disconnecting the two ends like that?''

``Finding the joining point isn't too difficult; you can just look for an abnormality in the chain. The hard part is actually disconnecting them without triggering the notification; as doing so requires precision. Even connecting the charm into a full circle is a huge feat.''

``That's almost impossible, much less doing it the night of your escape.''

``If they dare to use forbidden magic like that,'' pointed out the acquaintance, ``they wouldn't hesitate to kill a person of anything less than pure blood. This is a dangerous place to meet; we should gather a short distance from here. A sleeping person can't track you, of course. And for this purpose, I brought something.'' Telto pulled a knotted rope out of a sack. ``Tie this to one of the pillars out there and climb down. And take a bag too.''

``Thank you very much.''

``So you can read and write, is that correct? Do you enjoy it?''

``I can. I used to enjoy writing, but now my imagination has run out. At the same time, I'd like more books to read.''

``I'd like to talk more, but morning is approaching, and I'd like to survive. I will see you next night.''

While Telto, carrying the lantern, leaped from the loggia, Myllada started stashing the rope and the bag into the closet and changing into the open dress, crawling into the bed before sleep caught her again.

\chapter{Third chapter}

When the old man Nasrelten opened the door to his daughter's room, he was puzzled to see a bucket of water inside.