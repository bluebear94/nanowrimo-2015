\chapter{The Case of the Night-Runner}
	
As the dim moonlight forspread through the ever-waking boughs of cedars rustling in a gentle lullaby among the placid wind, and the nearby creek, once a leviathan waterfall from the distant peaks, flowed as it always did, a young person with flowing lavender hair in open clothing sat on the birch-plank floor of the fifth-story loggia, enjoying a rare hour free of mundane housework. Myllada, no more than thirteen years of age at that time, once could fantasize of tales and legends, war and peace, the great and the small, that lay beyond the cedars, or even this village, and had even written of them, but now her imaginations had nearly run out. After all, speculating what is there does not compare to actually knowing what is there, nor does mere knowledge compare to seeing these places firsthand.

Myllada also held a long-standing interest over mathematics and nature, interests that her parents, allegedly desiring an ignoramus to perform chores and marry young, would not tolerate. Having seen no interior of a school or library, she had learned to read using a book she had received eight years ago from an eccentric instructor, stashed in a closet alongside seven score pages of her own writing, avoiding the wrath of the flame.

By then, this inquisitiveness had vexed the young person for no fewer than three thousand nights in the confines of the mansion uneroded by time, wherein one could not find any traces of modern society.

The unexpected clamor of bushes snapped her out of reverie. Myllada, who seldom had the opportunity of conversing with those outside her family, transitioned into a kneeling position, peering over the rails in order to discern who, or what, made that sound.

Flashing a cloak as dark as the rest of its outfit, a form scurried along the side of the building.

Loudly enough for the figure to hear the sound, but quietly enough to avoid waking up anyone else in the house, Myllada whistled.

The figure stopped and turned around, running to the source of the whistle.

Knowing that any speech would attract attention, Myllada brandished her hand, receded into her room, pulled out a sheet of paper and a pen, and scribbled a message before folding the paper several times and tossing it onto the needle-covered grounds, ensuring that the letter did not land on a lower loggia.

The note thudded on hitting the ground, at which the unidentified scampered to collect it, before bowing shallowly and making haste out of the forest.

As the surroundings of the house again became desolate, Myllada again felt the hopelessness of residing there. But with this cloud came a silver lining: the narrowest probability that whomever she spotted would respond to her plea; help her escape the wretched prison.

\centeredstars

As the sun rose, a person, having returned home not long ago, opened a note that flew from a mansion untouched by centuries past. Not only was the letter unsigned, but the handwriting flew in all directions, hardly organized, and a style unexpected from someone who would live in such a place.

\q{
	I have been stuck in this house for many years, without a day of school, being raised as an ignoramus whose only purpose is to maintain the house and marry young. You are one of the few people other than family members whom I can reach.
	
	Please consider replying by any feasible measures. If I find you helpful, then I will find something for you in return. Make sure to arrive at night and bring clothing.
}

The recipient wondered why the sender asked for clothing -- perhaps the sender was naked? In any case, there was work to be done, to which the message was completely irrelevant.

\centeredstars

By midday the excitement over the mysterious visitor has eroded, and the possibility of any further external contact seemed a distant dream. \emph{That person will never come; what do I have in return?} Myllada pondered as she swept the corridors, lapsing for a few seconds.

``A woman who does not sweep is worthless!'' came a deep yell behind her, shocking her back into the monotonous task.

\emph{I could count the people who use that word with one hand. Well, on the bright side, if I'm worthless, then it shouldn't matter much to get out of here!} Myllada humorously concluded while joining thought and movement into the same rhythm. \emph{But, of course, I can't just brazenly say ``I'm useless to you'' and frolic out through the door.}